\documentclass{beamer}
\usetheme{default}
\usepackage{listings}
\lstset{mathescape, basicstyle=\ttfamily}

\title{Lessons learnt from formalizing theoretical~computer~science}
\author{Martin Dvorak}


\begin{document}

\begin{frame}[plain]
\maketitle
\end{frame}

\begin{frame}
\frametitle{What is Lean}
Lean 4 is a powerful programming language

Emphasis on formal verification

Type system based on the Calculus of Inductive Constructions

Large library of formally-verified mathematics (over $10^5$ lemmas)
\end{frame}

\begin{frame}
\frametitle{Showcase in VS Code}
\pause
The rest is all about my mistakes
\pause (and what I learnt from them
\pause (sometimes))
\end{frame}

\begin{frame}
\frametitle{Trying to prove a statement that doesn't hold}
\pause
Yeah, duh!
\end{frame}

\begin{frame}
\frametitle{Grammars are closed under concatenation}
\pause
Construction:
\begin{align*} 
	G_1 &= ( N_1, T, P_1, S_1 ) \\ 
	G_2 &= ( N_2, T, P_2, S_2 ) \\
	G\; &= ( N_1 \mskip-2mu\cup\mskip-1mu N_2 \mskip-2mu\cup\mskip-1mu \{ S \},\;
	T,\;\, P_1 \mskip-2mu\cup\mskip-1mu P_2
	\mskip-1mu\cup \{ S \rightarrow S_1 S_2 \},\: S\mskip1mu )
\end{align*}
\pause
Counterexample:
\begin{align*}
	P_1 &= \{ S_1 \rightarrow S_1 a,\, S_1 \rightarrow \epsilon \} \\
	P_2 &= \{ S_2 \rightarrow S_2 a,\, S_2 \rightarrow \epsilon,\, a S_2 \rightarrow b \}
\end{align*}
\pause
We get:
$$ L_1 = L_2 = \{ a^n \,|\, n \in \mathbb{N}_0 \} = L_1 L_2 $$
\pause
However:
$$ S \Rightarrow S_1 S_2 \Rightarrow S_1 a\mskip1mu S_2 \Rightarrow S_1 b \Rightarrow b $$
\end{frame}

\begin{frame}[fragile]
\frametitle{Carelessly assuming union of assumptions}
\begin{lstlisting}
[OrderedCancelAddCommMonoid C]
lemma Function.HasMaxCutProperty.
    forbids_commutativeFractionalPolymorphism

[OrderedAddCommMonoidWithInfima C]
lemma FractionalOperation.
    IsFractionalPolymorphismFor.
    expressivePowerVCSP

[OrderedCancelAddCommMonoidWithInfima C]
theorem ValuedCSP.CanExpressMaxCut.
    forbids_commutativeFractionalPolymorphism
\end{lstlisting}
The two latter classes extend \texttt{[CompleteSemilatticeInf C]}.
\pause
The assumption is too strong!
\begin{lstlisting}
($\exists$ x y : C, x < y) $\rightarrow$ False
\end{lstlisting}
\end{frame}

\begin{frame}[fragile]
\frametitle{Not taking time to develop good notation}
We write $A\,x$ using a function \texttt{Matrix.mulVec}
\begin{lstlisting}
A.mulVec x
\end{lstlisting}
We write $x^\top\!A$ using a function \texttt{Matrix.vecMul}
\begin{lstlisting}
Matrix.vecMul x A
\end{lstlisting}
\pause
\bigskip
Now we have infix operators \texttt{*$_v$} and \texttt{$_v$*}
\begin{lstlisting}
A *$_v$ x
x $_v$* A
\end{lstlisting}
\end{frame}

\begin{frame}[fragile]
\frametitle{Not factoring out useful lemmas}
\begin{lstlisting}
lemma {x$_1$ x$_2$ z$_1$ z$_2$ : List T} {a$_1$ a$_2$ : T}
    (notin_x : a$_2$ $\notin$ x$_1$) (notin_z : a$_2$ $\notin$ z$_1$) :
  (x$_1$ ++ [a$_1$] ++ z$_1$) = (x$_2$ ++ [a$_2$] ++ z$_2$) $\iff$
    (x$_1$ = x$_2$) $\wedge$ (a$_1$ = a$_2$) $\wedge$ (z$_1$ = z$_2$)
\end{lstlisting}
\end{frame}

\begin{frame}[fragile]
\frametitle{Reïnventing the wheel}
Writing an existing definition from scratch
\begin{lstlisting}
List.count
\end{lstlisting}
\pause
Developing lemmas about it
\begin{lstlisting}
List.count_eq_zero
\end{lstlisting}
\pause
Not knowing existing theorems
\begin{lstlisting}
Classical.choose_spec
Finset.prod_erase_eq_div
\end{lstlisting}
\end{frame}

\begin{frame}[fragile]
\frametitle{Using too many ``collection types''}
\begin{lstlisting}
Fin n $\rightarrow$ T
Array
List
Multiset
Finset
Fintype
\end{lstlisting}
\end{frame}

\end{document}
